\documentclass[b5paper]{article}
\usepackage[b5paper,top=1cm,bottom=2cm,left=2cm,right=2cm,marginparwidth=1.75cm]{geometry}
\usepackage[english]{babel}
\usepackage[utf8x]{inputenc}
\usepackage[T1]{fontenc}
\usepackage{amsmath}
\usepackage{indentfirst}
\usepackage{multicol}

\linespread{1.3}

\begin{document}
\large\hfill \textbf{125} \\
\normalsize\date{}
\normalsize\pagestyle{empty}
\begin{flushleft}
[32] Roger B. Nelsen.
\emph{Proofs without Words II: More Exercises in Visual Thinking.}
Mathematical Association of America, Washington, DC, 2000.
[33] Robert A. Nelson and M. G. Olsson. The pendulum: Rich physics from a simple
system.\emph{American Journal of Physics,}  54(2):112–121, 1986.

[34] R. C. Pankhurst. \emph{Dimensional Analysis and Scale Factors.} Chapman and Hall, London,
1964.

[35] George Polya. \emph{Induction and Analogy in Mathematics,} volume 1 of \emph{Mathematics and Plausible Reasoning.} Princeton University Press, Princeton, New Jersey, 1954.

[36] George Polya. \emph{Patterns of Plausible Inference, }volume 2 of\emph{ Mathematics and Plausible Reasoning.v} Princeton University Press, Princeton, New Jersey, 1954.

[37] George Polya. \emph{How to Solve It: A New Aspect of the Mathematical Method. }Princeton University Press, Princeton, New Jersey, 1957/2004.

[38] Edward M. Purcell. Life at low Reynolds number. \emph{American Journal of Physics,
}45(1):3–11, 1977.

[39] Gilbert Ryle. \emph{The Concept of Mind.} Hutchinson’s University Library, London, 1949.

[40] Carl Sagan.\emph{Contact.}  Simon $\&$ Schuster, New York, 1985.

[41] E. Salamin. Computation of pi using arithmetic-geometric mean.\emph{ Mathematics of
Computation,} 30:565–570, 1976.

[42] Dava Sobel.\emph{ Longitude: The True Story of a Lone Genius Who Solved the Greatest
Scientific Problem of His Time. Walker and Company,} New York, 1995.

[43] Richard M. Stallman and Gerald J. Sussman. Forward reasoning and dependencydirected
backtracking in a system for computer-aided circuit analysis. AI Memos
380, MIT, Artificial Intelligence Laboratory, 1976.

[44] Edwin F. Taylor and John Archibald Wheeler.\emph{ Spacetime Physics: Introduction to
Special Relativity.} W. H. Freeman, New York, 2nd edition, 1992.

[45] Silvanus P. Thompson.\emph{ Calculus Made Easy: Being a Very-Simplest Introduction to
Those Beautiful Methods of Reasoning Which are Generally Called by the Terrifying
Names of the Differential Calculus and the Integral Calculus.} Macmillan, New York,
2nd edition, 1914.

[46] D. J. Tritton.\emph{ Physical Fluid Dynamics.} Oxford University Press, New York, 2nd
edition, 1988.

[47] US Bureau of the Census.\emph{ Statistical Abstracts of the United States: 1992. GovernmentPrinting Office, Washington, DC, 112th edition,} 1992.

[48] Max Wertheimer.\emph{ Productive Thinking. }Harper, New York, enlarged edition, 1959.

[49] Paul Zeitz. \emph{ the Art and Craft of Problem Solving.} Wiley, Hoboken, New Jersey, 2nd edition, 2007.
\end{flushleft}

\newpage
\mbox{}
\newpage
~\\~\\
 \Huge\textbf{Index}
~\\~\\
~\\~\\~\\
~\\
\normalsize\emph An italic page number refers to a problem on that page~\\~\\
\setlength{\columnsep}{1cm}
\begin{multicols}{2}
$\nu$
  \par see kinematic viscosity
1 or few \par
see few\par
(approximately equal) 6
$\pi$, computing\par
arctangent series 64\par
Brent–Salamin algorithm {}65
∝ (proportional to) 6\quad\quad\quad\quad\quad\quad
$\sim$ (twiddle) 6, 44\quad\quad
$\omega$\par
see angular frequency\par\quad\quad\quad\quad
\par{} \par analogy, reasoning by 99–121\par
dividing space with planes 103–107\par
generating conjectures\par
 \par see conjectures: gen\par\par
operators 107–113\par
left shift (L) 108–109
summation  109\par
preserving crucial features 100, 118,
\quad120\par
pyramid volume 19\par
spatial angles 99–103\par
tangent-root sum 118–121\par
testing conjectures\par
\quad \quad see conjectures: testing\par
to polynomials 118–121\par
transforming dependent variable 101
angles, spatial 99–103\par
angular frequency 44\par
Aristotle xiv\par
arithmetic–geometric mean 65\par


arithmetic-mean–geometric-mean in-\par equality
60–66\par
applications 63–66\par
computing $\pi$64–66\par\quad
maxima 63–64\par\quad
equality condition 62\par
numerical examples 60\par
pictorial proof 61–63\par
symbolic proof 61\quad \quad
arithmetic mean\par
see also geometric mean\par
picture for 62\par indent\quad
asymptotes of tan x 114\quad
atmospheric pressure 34\quad
back-of-the-envelope estimates\par
correcting 78\par
mental multiplication in 77\par
minimal accuracy required for 78\par
powers of 10 in 78\quad
balancing 41\quad
Basel sum (\quad
n−2) 76, 113, 116, 121\quad
beta function 98\quad
big part, correcting \par
see also taking out the big part\par
additive messier than multiplicative\par
corrections 80\quad
using multiplicative corrections\par
see fractional changes\par
using one or few 78\quad
big part, taking out\par
see taking out the big part\par
\end{multicols}

\newpage
 \large\textbf{128} \hfill
 \linespread{1.1}
 ~\\
 \normalsize
\setlength{\columnsep}{1cm}
\begin{multicols}{2}


binomial coefficients 96, 107\par
binomial distribution 98\quad\quad
binomial theorem 90, 97\par
bisecting a triangle 70–73\quad\quad
bits, CD capacity in 78\par
blackbody radiation 87\par
boundary layers 27\par\quad
brain evolution 57\par
Buckingham, Edgar 26\par
~\\
calculus, fundamental idea of 31\par
CD-ROM\par\quad
see also CD\par\quad
same format as CD 77\par
CD/CD-ROM, storage capacity 77–79\par
characteristic magnitudes (typical magni\quad tudes)\par
characteristic times 44\par
checking units 78\par\quad
\quad area from circumference 76\par
\quad as polygon with many sides 72\par
comparisons, nonsense with different\par
\quad dimensions 2\par
cone free-fall distance 35\par
cone templates 21\par
conical pendulum 48\par
conjectures\par\quad
discarding coincidences 105, 119\par\quad
explaining 119\par
generating 100, 103, 104, 105\par
probabilities of 105\par\quad
testing 100, 101, 104, 106, 111, 119\par\quad
getting more data 100, 105, 106\par
constants of proportionality\par
Stefan–Boltzmann constant 11\par
constraint propagation 5\par
contradictions 20\par
convergence, accelerating 65, 68\par\quad
convexity 104\par
copyright raising book prices 82\par
Corfield, David 105\par
cosine\par\quad
integral of high power 94–97\par
small-angle approximation\par\quad
derived 86\par

\quad used 95\par
cube, bisecting 73\par\quad\quad
~\\
d (differential symbol) 10, 43\par\quad
degeneracies 103\par
derivative as a ratio 38\par
derivatives\par\quad
approximating with nonzero  40\par
secant approximation 38\par\quad
errors in 39\par\quad
improved starting point 39\par\quad
large error 38\par
vertical translation 39\par
dimensions of 38\par\quad\quad\quad
secant approximation to 38\par
significant-change approximation
acceleration 43\par
Navier–Stokes derivatives 45\par
scale and translation invariance 40\par
translation invariance 40\par\quad\quad\quad
desert-island method 32\par\quad\quad
differential equations\par
checking dimensions 42\par\quad
linearizing 47, 51–54\par\quad
orbital motion 12\par\quad
pendulum 46\par\quad
simplifying into algebraic equations\par
43–46\par\quad
spring–mass system 42–45\par\quad
exact solution 45\par\quad
pendulum equation 47\par
dimensional analysis\par
see dimensions, method of; dimensionless\par
groups\par\quad
dimensionless constant\par
Gaussian integral 10\par
simple harmonic motion 48\par
Stefan–Boltzmann law 11\par
dimensionless groups 24\par\quad\quad
drag 25\par\quad
free-fall speed 24\par\quad
pendulum period 48\par
spring–mass system 48\par
\end{multicols}
\end{document}
